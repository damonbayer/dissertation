\chapter{Background}
\label{ch:background}

\section{Mathematical Models for the Spread of Infectious Diseases}
\label{sec:math_models}
% 10-20 pages
\subsection{Compartmental Models}

\subsubsection{SIR Model}


A popular tool used for these tasks is the compartmental model.
The most basic and prominent of these is the Susceptible-Infected-Removed (SIR) model.
In these models, the population of interest is divided into compartments (e.g., susceptible, infectious, and recovered), and individuals transition between compartments at a specified rate (e.g. transitions from infectious compartment to recovered compartment happen at a rate proportional to the number of infectious individuals).
Overviews of mechanistic compartmental models for disease dynamics can be found in \citep{anderson1992infectious, Brauer2008, keeling2011modeling, 10.1093/aje/kww021}.
These models can be represented deterministically or stochastically.
The stochastic models are more computationally difficult to fit and are particularly advantageous when there are few infections, as the beginning or end of an outbreak.
In these scenarios, the randomness involved at the subject level can have a major impact on the course of the outbreak.
In the most extreme case, one infectious individual could be introduced to a population, but, by chance, not infect anyone else.
In contrast, deterministic models are more computationally feasible and tend to work well when all compartments are suitably large \citep{doi:10.1098/rspb.2015.0347}.
This work is only concerned with these deterministic models in large population settings.

\begin{figure}
    \centering
% https://q.uiver.app/?q=WzAsMyxbMCwwLCJTIl0sWzEsMCwiSSJdLFsyLDAsIlIiXSxbMCwxXSxbMSwyXV0=
\begin{tikzcd}
	S & I & R
	\arrow[from=1-1, to=1-2]
	\arrow[from=1-2, to=1-3]
\end{tikzcd}
    \caption{Caption}
    \label{fig:SIR_diagram}
\end{figure}


\subsubsection{SEIRDS Extension to the SIR Model}

\begin{figure}
    \centering
% https://q.uiver.app/?q=WzAsNSxbMCwxLCJTIl0sWzQsMSwiSSJdLFs1LDAsIlIiXSxbMiwxLCJFIl0sWzUsMiwiRCJdLFsxLDJdLFswLDNdLFszLDFdLFsxLDRdLFsyLDAsIiIsMSx7ImN1cnZlIjo1fV1d
\begin{tikzcd}[column sep=scriptsize]
	&&&&& R \\
	S && E && I \\
	&&&&& D
	\arrow[from=2-5, to=1-6]
	\arrow[from=2-1, to=2-3]
	\arrow[from=2-3, to=2-5]
	\arrow[from=2-5, to=3-6]
	\arrow[curve={height=30pt}, from=1-6, to=2-1]
\end{tikzcd}
    \caption{Caption}
    \label{fig:SEIRDS_diagram}
\end{figure}

\subsection{Stochastic models}


\subsection{Logistic growth of variants}


\section{Bayesian Inference and Markov Chain Monte Carlo}
\label{sec:bayesian-mcmc}
% 8-10 pages
\begin{itemize}
    \item What is Bayesian inference?
    \item Describe Hamiltonian Monte Carlo (reference neal mcmc handbook http://www.mcmchandbook.net/HandbookChapter5.pdf)
    \item Describe Markov chain Monte Carlo (reference neal mcmc handbook http://www.mcmchandbook.net/HandbookChapter5.pdf) 
    \item Describe No-U-Turn Sampler (https://www.jmlr.org/papers/volume15/hoffman14a/hoffman14a.pdf)
\end{itemize}

\section{Forecast Assessment}
\label{sec:forecasting_techniques_and_assessment}

\begin{itemize}
\item https://epiforecasts.io/scoringutils/
\item https://www.jstatsoft.org/article/view/v090i12
\item Maybe cite some theory from Gneiting
\item Maybe cite some theory from Reich papers 
\item CRPS paper Krueger, F., Lerch, S., Thorarinsdottir, T.L. and T. Gneiting (2021): ‘Predictive inference based on Markov chain Monte Carlo output’, International Statistical Review 89, 274-301. doi:10.1111/insr.12405
\item scoringRules paper https://www.jstatsoft.org/article/view/v090i12
\item proper scoring rules

\section{Fiducial Inference}

\end{itemize}

% 8-10 pages