\thesistitle{Inference and Forecasting Using Infectious Disease Surveillance Data}

%"Dissertation" for PhD, "Thesis" for master's
\documenttitle{Dissertation}

\degreename{Doctor of Philosophy}

% Use the wording given in the official list of degrees awarded by UCI:
% http://www.rgs.uci.edu/grad/academic/degrees_offered.htm
\degreefield{Statistics}

% Your name as it appears on official UCI records.
\authorname{Damon Bayer}

% Use the full name of each committee member and full title 
% (e.g. Professor/Associate Professor).
\committeechair{Professor Volodymyr Minin}
\othercommitteemembers
{
  Professor Veronica Berrocal\\
  Associate Professor Weining Shen\\
  Assistant Professor Daniel Parker
}

\degreeyear{2023}

\copyrightdeclaration
{
  {\copyright} {\Degreeyear} \Authorname
}

% If you have previously published parts of your manuscript, you must list the
% copyright holders; see Section 3.2 of the UCI Thesis and Dissertation Manual.
% Otherwise, this section may be omitted.
\prepublishedcopyrightdeclaration
{
Portions of Chapter~\ref{ch:content_1} {\copyright} 2023 John Wiley \& Sons Ltd\\
All other materials {\copyright} {\Degreeyear} \Authorname
}

% The dedication page is optional
% (comment out to exclude).
% \dedications
% {
%   (Optional dedication page)
  
%   To ...
% }

\acknowledgments
{
  % (You must acknowledge grants and other funding assistance. 
  
  % You may also acknowledge the contributions of professors and
  % friends.
  
  % You also need to acknowledge any publishers of your previous
  % work who have given you permission to incorporate that work
  % into your dissertation. See Section 3.2 of the UCI Thesis and
  % Dissertation Manual.)

Optional acknowledgements tk.
% \begin{itemize}
%     \item Volodymyr Minin
%     \item NIAID Mentors
%     \begin{itemize}
%         \item Jon Fintzi
%         \item Mike Fay
%     \end{itemize}
%     \item UCI Colleagues
%     \begin{itemize}
%         \item Isaac Goldstein for starting Minin Group Irvine, learning Julia with me, and pickleball
%         \item Rest of Minin Group for intellectual stimulation and juicy gossip
%         \item Volleyball crew
%     \end{itemize}
%     \item Olivia Hogan-Stark
% \end{itemize}

This work was supported by funding from the UCI Infectious Disease Science Initiative, the UC CDPH Modeling Consortium, and NIH grant R01AI147336.

This work utilized the infrastructure for high-performance and high-throughput computing, research data storage and analysis, and scientific software tool integration built, operated, and updated by the Research Cyberinfrastructure Center (RCIC) at the University of California, Irvine.

The text of Chapter 3 of this dissertation is a modified reprint of the material as it appears in \textit{Statistics in Medicine} \citep{Bayer2023Confidence}, used with permission from Wiley.


% Required
% \begin{itemize}
%         \item Ch 3
%         \item For sharing the data from the Kalish, et al study, we thank Matthew Memoli and the LID Clinical Studies Unit of the National Institute of Allergy and Infectious Diseases, NIH, Kaitlyn Sadtler from the National Institute of Biomedical Imaging and Bioengineering, NIH, Matthew Hall from the National Center for Advancing Translational Sciences, NIH, and Dominic Esposito from the Fredrick National Laboratory for Cancer Research, NCI, NIH.
%         \item Thank Wiley for allowing me to incorporate into dissertation
%         \item Ch 4
%         \item This work utilized the infrastructure for high-performance and high-throughput computing, research data storage and analysis, and scientific software tool integration built, operated, and updated by the Research Cyberinfrastructure Center (RCIC) at the University of California, Irvine.
%         \item We are grateful for funding from the UCI Infectious Disease Science Initiative.
%         \item This work was made possible in part through support from the UC CDPH Modeling Consortium. D.B, I.H.G, and V.M.M were supported in part by NIH grant R01AI147336. V.M.M was supported in part by NIH grant R01AI170204 and NSF grant DMS 1936833. ER was supported by the Division of Intramural Research, NIAID, NIH. This project has been funded in part with federal funds from the National Cancer Institute, National Institutes of Health, under Contract No. 75N91019D00024, Task Order No. 75N91019F00130. This work was in part supported by the intramural research programs of the National Institutes of Health, Bethesda, MD. The content of this publication does not necessarily reflect the views or policies of the Department of Health and Human Services, nor does mention of trade names, commercial products, or organizations imply endorsement by the U.S. Government.
% \end{itemize}
}


% Some custom commands for your list of publications and software.
\newcommand{\mypubentry}[3]{
  \begin{tabular*}{1\textwidth}{@{\extracolsep{\fill}}p{4.5in}r}
    \textbf{#1} & \textbf{#2} \\ 
    \multicolumn{2}{@{\extracolsep{\fill}}p{.95\textwidth}}{#3}\vspace{6pt} \\
  \end{tabular*}
}
\newcommand{\mysoftentry}[3]{
  \begin{tabular*}{1\textwidth}{@{\extracolsep{\fill}}lr}
    \textbf{#1} & \url{#2} \\
    \multicolumn{2}{@{\extracolsep{\fill}}p{.95\textwidth}}
    {\emph{#3}}\vspace{-6pt} \\
  \end{tabular*}
}

% Include, at minimum, a listing of your degrees and educational
% achievements with dates and the school where the degrees were
% earned. This should include the degree currently being
% attained. Other than that it's mostly up to you what to include here
% and how to format it, below is just an example.
%
% CV is required for PhD theses, but not Master's
% comment out to exclude
\curriculumvitae
{

\textbf{EDUCATION}
  
  \begin{tabular*}{1\textwidth}{@{\extracolsep{\fill}}lr}
    \textbf{Doctor of Philosophy in Statistics} & \textbf{2023} \\
    \vspace{6pt}
    University of California, Irvine & \emph{Irvine, CA} \\
    \textbf{Master of Science in Statistics} & \textbf{2020} \\
    \vspace{6pt}
    University of California, Irvine & \emph{Irvine, CA} \\
    \textbf{Master of Science in Mathematics} & \textbf{2018} \\
    \vspace{6pt}
    South Dakota State University & \emph{Brookings, SD} \\
    \textbf{Bachelor of Science in Mathematics} & \textbf{2016} \\
    \vspace{6pt}
    South Dakota State University & \emph{Brookings, SD} \\
  \end{tabular*}

% \vspace{12pt}
% \textbf{RESEARCH EXPERIENCE}

%   \begin{tabular*}{1\textwidth}{@{\extracolsep{\fill}}lr}
%     \textbf{Graduate Research Assistant} & \textbf{2007--2012} \\
%     \vspace{6pt}
%     University of California, Irvine & \emph{Irvine, California} \\
%   \end{tabular*}

% \vspace{12pt}
% \textbf{TEACHING EXPERIENCE}

%   \begin{tabular*}{1\textwidth}{@{\extracolsep{\fill}}lr}
%     \textbf{Teaching Assistant} & \textbf{2009--2010} \\
%     \vspace{6pt}
%     University name & \emph{City, State} \\
%   \end{tabular*}

% \pagebreak

% \textbf{REFEREED JOURNAL PUBLICATIONS}

%   \mypubentry{Ground-breaking article}{2012}{Journal name}

% \vspace{12pt}
% \textbf{REFEREED CONFERENCE PUBLICATIONS}

%   \mypubentry{Awesome paper}{Jun 2011}{Conference name}
%   \mypubentry{Another awesome paper}{Aug 2012}{Conference name}

% \vspace{12pt}
% \textbf{SOFTWARE}

%   \mysoftentry{Magical tool}{http://your.url.here/}
%   {C++ algorithm that solves TSP in polynomial time.}

}

% The abstract was previously limited to a maximum of 350 words, 
% but the UCI manual at https://etd.lib.uci.edu/electronic/td2e#2.2.1.
% currently does not indicate that there is any word limit for the abstract
\thesisabstract
{
Statistical modeling of infectious disease data is among the oldest applications of statistics. Today, it is an increasingly relevant application of research, due to globalization that enables diseases to spread further and faster, as well as the abundance of relevant data from electronic surveillance systems, seroprevalence studies, and genetic sequencing of pathogens. In this work, we develop novel statistical methods to combine varied data sources to improve both inference and forecasting. First, we work with data from assay validation studies and active surveillance studies to develop confidence intervals for prevalence estimates from complex surveys with imperfect assays. In this complicated setting, there are no established competitive methods, and ours exhibits at least nominal coverage. In addition, we apply our model in simplified cases where competitors exist and demonstrate desirable properties. Next, we develop a semi-parametric Bayesian compartmental model that effectively integrates passively collected time series of diagnostic tests and mortality data, as well as actively collected seroprevalence data. We emphasize retrospective inference and evaluate the utility of each data stream in the context of short-term forecasting. Finally, we focus on healthcare demand forecasting during epidemic surges of pathogen variants capable of immune escape. We build upon our Bayesian compartmental model to incorporate time series of cases, hospitalizations, ICU admissions, deaths, and genetic sequence counts. We show that using genetic information leads to superior forecasting performance, compared to traditional models. Throughout each project, we employ our methods to analyze a variety of COVID-19 data sets at the county, state, and national levels.
}


%%% Local Variables: ***
%%% mode: latex ***
%%% TeX-master: "thesis.tex" ***
%%% End: ***
