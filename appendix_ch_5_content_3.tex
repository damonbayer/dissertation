\chapter{Additional Material for Chapter 5}
\graphicspath{{figures/ch_5/}}
\addtocontents{toc}{\protect\setcounter{tocdepth}{0}}

\section{Simulation model}
\label{ch_5:sec:full_simulation_model_explanation}
\addtocontents{toc}{\protect\setcounter{tocdepth}{2}}

\section{Additional Figures}

\begin{figure}
    \centering
    \includegraphics[width=1.0\columnwidth]{simulated_binned_data_slow_plot}
    \caption[Simulated data set for the slow takeover speed scenario.]{Simulated data set for the slow takeover speed scenario.
    The gray shaded areas indicate the time points for which we create forecasts}
    \label{ch_5:fig:simulated_binned_data_slow_plot}
\end{figure}

\begin{figure}
    \centering
    \includegraphics[width=1.0\columnwidth]{simulated_binned_data_fast_plot}
    \caption[Simulated data set for the fast takeover speed scenario.]{Simulated data set for the fast takeover speed scenario.
    The gray shaded areas indicate the time points for which we create forecasts}
    \label{ch_5:fig:simulated_binned_data_fast_plot}
\end{figure}

\section{Additional simulation study results for cases, ICU occupancy, and deaths}
\label{ch_5:sec:sim_cases_icu_death}

\begin{figure}
    \centering
    \includegraphics[width=1.0\columnwidth]{simulated_forecast_comparison_data_hospitalizations_slow_plot}
\caption[Hospital occupancy forecasts for simulated slow takeover speed data.]{Hospital occupancy forecasts from three models at 1, 2, and 4-week forecast horizons for the simulated slow takeover speed data.}
    \label{ch_5:fig:simulated_forecast_comparison_data_hospitalizations_slow_plot}
\end{figure}

\begin{figure}
    \centering
    \includegraphics[width=1.0\columnwidth]{simulated_forecast_comparison_data_hospitalizations_fast_plot}
\caption[Hospital occupancy forecasts for simulated fast takeover speed data.]{Hospital occupancy forecasts from three models at 1, 2, and 4-week forecast horizons for the simulated fast takeover speed data.}
    \label{ch_5:fig:simulated_forecast_comparison_data_hospitalizations_fast_plot}
\end{figure}

\begin{figure}
    \centering
    \includegraphics[width=1.0\columnwidth]{simulated_crps_comparison_dotplot_data_new_cases_plot}
    \caption{CRPS summaries for new cases forecasts at 1, 2, and 4-week horizons for three simulated data sets. Lower CRPS is better.}
    \label{ch_5:fig:simulated_crps_comparison_dotplot_data_new_cases_plot}
\end{figure}

\begin{figure}
    \centering
    \includegraphics[width=1.0\columnwidth]{simulated_crps_comparison_dotplot_data_icu_plot}
    \caption{CRPS summaries for ICU occupancy forecasts at 1, 2, and 4-week horizons for three simulated data sets. Lower CRPS is better.}
    \label{ch_5:fig:simulated_crps_comparison_dotplot_data_icu_plot}
\end{figure}

\begin{figure}
    \centering
    \includegraphics[width=1.0\columnwidth]{simulated_crps_comparison_dotplot_data_new_deaths_plot}
    \caption{CRPS summaries for new deaths forecasts at 1, 2, and 4-week horizons for three simulated data sets. Lower CRPS is better.}
    \label{ch_5:fig:simulated_crps_comparison_dotplot_data_new_deaths_plot}
\end{figure}

\begin{figure}
    \centering
    \includegraphics[width=1.0\columnwidth]{simulated_crps_comparison_data_new_cases_plot}
    \caption{Individual CRPS for new cases forecasts at 1, 2, and 4-week horizons for three simulated data sets. Lower CRPS is better.}
    \label{ch_5:fig:simulated_crps_comparison_data_new_cases_plot}
\end{figure}

\begin{figure}
    \centering
    \includegraphics[width=1.0\columnwidth]{simulated_crps_comparison_data_hospitalizations_plot}
    \caption{Individual CRPS for hospital occupancy forecasts at 1, 2, and 4-week horizons for three simulated data sets. Lower CRPS is better.}
    \label{ch_5:fig:simulated_crps_comparison_data_hospitalizations_plot}
\end{figure}

\begin{figure}
    \centering
    \includegraphics[width=1.0\columnwidth]{simulated_crps_comparison_data_icu_plot}
    \caption{Individual CRPS for ICU occupancy forecasts at 1, 2, and 4-week horizons for three simulated data sets. Lower CRPS is better.}
    \label{ch_5:fig:simulated_crps_comparison_data_icu_plot}
\end{figure}

\begin{figure}
    \centering
    \includegraphics[width=1.0\columnwidth]{simulated_crps_comparison_data_new_deaths_plot}
    \caption{Individual CRPS for new deaths forecasts at 1, 2, and 4-week horizons for three simulated data sets. Lower CRPS is better.}
    \label{ch_5:fig:simulated_crps_comparison_data_new_deaths_plot}
\end{figure}

\begin{figure}
    \centering
    \includegraphics[width=1.0\columnwidth]{simulated_peak_crps_plot}
    \caption{Caption}
    \label{ch_5:fig:simulated_peak_crps_plot}
\end{figure}

\section{Simulation study sensitivity analysis}
\label{ch_5:sec:sim_sensitivity}

\begin{figure}
    \centering
    \includegraphics[width=0.75\columnwidth]{sensitivity_simulated_crps_comparison_dotplot_data_hospitalizations_plot}
    \caption[CRPS summaries for hospital occupancy forecasts for simulated data sets.]{CRPS summaries for hospital occupancy forecasts at 1, 2, and 4-week horizons for three simulated data sets. Lower CRPS is better.}
    \label{ch_5:fig:sensitivity_simulated_crps_comparison_dotplot_data_hospitalizations_plot}
\end{figure}


\begin{figure}
    \centering
    \includegraphics[width=1.0\columnwidth]{sensitivity_simulated_peak_assessment_time_plot}
    \caption[Posterior predictive intervals for peak hospital occupancy timing for simulated data sets.]{Posterior predictive intervals for the time at which hospital occupancy reaches its maximum in three simulated data sets.
    Dots indicate the median of the predictive distribution, while the thick and thin lines represent central 80\% and 95\% intervals, respectively.
    Horizontal and vertical dashed lines indicate the true peak hospitalization time.}
    \label{ch_5:fig:sensitivity_simulated_peak_assessment_time_plot}
\end{figure}

\begin{figure}
    \centering
    \includegraphics[width=1.0\columnwidth]{sensitivity_simulated_peak_assessment_value_plot}
    \caption[Posterior predictive intervals for peak hospital occupancy for simulated data sets.]{Posterior predictive intervals for the maximum hospital occupancy in three simulated data sets.
    Dots indicate the median of the predictive distribution, while the thick and thin lines represent central 80\% and 95\% intervals, respectively.
    Horizontal dashed lines indicate the true peak hospital occupancy, while the vertical dashed lines indicate the true peak hospital occupancy time.}
    \label{ch_5:fig:sensitivity_simulated_peak_assessment_value_plot}
\end{figure}

\begin{figure}
    \centering
    \includegraphics[width=1.0\columnwidth]{sensitivity_simulated_peak_crps_dotplot_plot}
    \caption[CRPS summaries for peak hospital occupancy in simulated data sets.]{CRPS summaries for peak hospital occupancy timing and size for three simulated data sets. Lower CRPS is better.}
    \label{ch_5:fig:sensitivity_simulated_peak_crps_dotplot_plot}
\end{figure}

\section{Additional California data results}

\begin{figure}
    \centering
    \includegraphics[width=1.0\columnwidth]{real_data_crps_comparison_dotplot_data_new_cases_plot}
    \caption[CRPS summaries for new cases forecasts for real data sets.]{CRPS summaries for new cases forecasts at 1, 2, and 4-week horizons for California and Orange County data sets. Lower CRPS is better.}
    \label{ch_5:fig:real_data_crps_comparison_dotplot_data_new_cases_plot}
\end{figure}

\begin{figure}
    \centering
    \includegraphics[width=1.0\columnwidth]{real_data_crps_comparison_dotplot_data_icu_plot}
    \caption[CRPS summaries for ICU occupancy forecasts for real data sets.]{CRPS summaries for ICU occupancy forecasts at 1, 2, and 4-week horizons for California and Orange County data sets. Lower CRPS is better.}
    \label{ch_5:fig:real_data_crps_comparison_dotplot_data_icu_plot}
\end{figure}

\begin{figure}
    \centering
    \includegraphics[width=1.0\columnwidth]{real_data_crps_comparison_dotplot_data_new_deaths_plot}
    \caption[CRPS summaries for new deaths occupancy forecasts for real data sets.]{CRPS summaries for new deaths occupancy forecasts at 1, 2, and 4-week horizons for California and Orange County data sets. Lower CRPS is better.}
    \label{ch_5:fig:real_data_crps_comparison_dotplot_data_new_deaths_plot}
\end{figure}

\begin{figure}
    \centering
    \includegraphics[width=1.0\columnwidth]{real_data_crps_comparison_data_new_cases_plot}
    \caption{Individual CRPS for new cases forecasts at 1, 2, and 4-week horizons for California and Orange County data sets. Lower CRPS is better.}
    \label{ch_5:fig:real_data_crps_comparison_data_new_cases_plot}
\end{figure}

\begin{figure}
    \centering
    \includegraphics[width=1.0\columnwidth]{real_data_crps_comparison_data_hospitalizations_plot}
    \caption{Individual CRPS for hospital occupancy forecasts at 1, 2, and 4-week horizons for California and Orange County data sets. Lower CRPS is better.}
    \label{ch_5:fig:real_data_crps_comparison_data_hospitalizations_plot}
\end{figure}

\begin{figure}
    \centering
    \includegraphics[width=1.0\columnwidth]{real_data_crps_comparison_data_icu_plot}
    \caption{Individual CRPS for ICU occupancy forecasts at 1, 2, and 4-week horizons for California and Orange County data sets. Lower CRPS is better.}
    \label{ch_5:fig:real_data_crps_comparison_data_icu_plot}
\end{figure}

\begin{figure}
    \centering
    \includegraphics[width=1.0\columnwidth]{real_data_crps_comparison_data_new_deaths_plot}
    \caption{Individual CRPS for new deaths forecasts at 1, 2, and 4-week horizons for California and Orange County data sets. Lower CRPS is better.}
    \label{ch_5:fig:real_data_crps_comparison_data_new_deaths_plot}
\end{figure}

\begin{figure}
    \centering
    \includegraphics[width=1.0\columnwidth]{real_data_peak_crps_plot}
    \caption[Individual CRPS for peak hospital occupancy in real data sets.]{Individual CRPS for peak hospital occupancy timing and size in California and Orange County data sets. Lower CRPS is better.}
    \label{ch_5:fig:real_data_peak_crps_plot}
\end{figure}

\label{ch_5:sec:real_cases_icu_death}