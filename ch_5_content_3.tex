\chapter{Forecasting epidemic surges of competing pathogen variants}
\label{ch:content_3}

\section{Introduction}
\label{ch_5:sec:intro}

\begin{itemize}
    \item Forecasting in Epidemic modeling is important.
    \item A lot of forecasting methods have been tried recently.
    \item COVID-19 saw changing dynamics due to new variants
    \item When new variants arise, they can lead to waves, either due to increased transmissibility or immune escape
    \item We would like to incorporate this data into our models in a simple, but flexible way.
    \item Multi-strain models have been proposed, but are rarely (never?) fit to data.
    \item We base our forecasting on mechanistic modeling.
    \item We assess our models in a simulation study with different takeover speeds
    \item We apply to real data from Orange County, CA and the entire state of CA.
    \item We show our method is superior at forecasting and esp at forecasting peak timing of hosp (something healthcare people care a lot about) compared to a typical semi-parametric model.
\end{itemize}

\section{Methods}
\label{ch_5:sec:methods}

\subsection{Data}
\label{ch_5:subsec:data}


\begin{itemize}
    \item Data is from CDPH and GISAID at the county level. We also aggregate at the state level.
    \item We have time series of daily
    \begin{itemize}
        \item cases
        \item hospitalizations
        \item ICU
        \item deaths
        \item counts of sequences of variant 1
        \item counts of sequences of variant 2
    \end{itemize}
    \item in practice, how to determine variant 1 vs variant 2 is not obvious
    \item we lump non-genetic data in weekly bins to make less noisy and avoid weekend effects
    \item plot of the data for CA (and OC?)
    
\end{itemize}

\subsection{Transmission model}
\label{ch_5:subsec:transmission}


\begin{itemize}
    \item explain compartments
    \item explain system of ODE's
\end{itemize}

Model Diagram:

\begin{figure}
    \centering
% https://q.uiver.app/?q=WzAsNyxbMCwyLCJTIl0sWzEsMiwiRSJdLFsyLDIsIkkiXSxbNCwyLCJSIl0sWzQsMCwiRCJdLFsyLDEsIkgiXSxbMiwwLCJJQ1UiXSxbMCwxXSxbMSwyXSxbMiwzXSxbMywwLCIiLDAseyJjdXJ2ZSI6LTV9XSxbNSwzXSxbNiwzXSxbNiw0XSxbNSw2XSxbMiw1XV0=
\begin{tikzcd}[sep=scriptsize]
	&& ICU && D \\
	&& H \\
	S & E & I && R
	\arrow[from=3-1, to=3-2]
	\arrow[from=3-2, to=3-3]
	\arrow[from=3-3, to=3-5]
	\arrow[curve={height=-30pt}, from=3-5, to=3-1]
	\arrow[from=2-3, to=3-5]
	\arrow[from=1-3, to=3-5]
	\arrow[from=1-3, to=1-5]
	\arrow[from=2-3, to=1-3]
	\arrow[from=3-3, to=2-3]
\end{tikzcd}
    \caption{Caption}
    \label{ch_5:fig:model_diagram}
\end{figure}


\subsection{Surveillance model}
\label{ch_5:subsec:surveillance}

\begin{itemize}
    \item We connect the observed data to via emission distributions
\end{itemize}

\subsection{Bayesian model}
\label{ch_5:subsec:bayesian}

\begin{itemize}
    \item assume independence
    \item GMRF
    \item \( R_0(t) \) can be constant, GMRF.
    \item \( \omega(t) \) can be constant, GMRF or seq-informed.
    \item In principle, any parameters could be seq-informed, if you had some idea about a plausible reason.
\end{itemize}

\section{Results}
\label{ch_5:sec:results}

\subsection{Simulation study}
\label{ch_5:subsec:simulation}

\begin{itemize}
    \item We simulate from a two strain model
    \item 
\end{itemize}

\begin{figure}
    \centering
% https://q.uiver.app/?q=WzAsMTMsWzAsMCwiU18wIl0sWzEsMCwiRV8xIl0sWzIsMCwiSV8xIl0sWzIsMSwiUl8xIl0sWzEsMiwiRV97Mn0iXSxbMiwyLCJJXzIiXSxbMiwzLCJSXzIiXSxbMCwyLCJTXzIiXSxbMywwLCJIXzEiXSxbNCwwLCJJQ1VfMSJdLFszLDIsIkhfMiJdLFs0LDIsIklDVV8yIl0sWzQsMSwiRCJdLFswLDFdLFsxLDJdLFsyLDNdLFswLDRdLFs0LDVdLFszLDBdLFszLDRdLFs1LDZdLFs2LDddLFsyLDhdLFs4LDldLFs5LDNdLFs4LDNdLFs1LDEwXSxbMTAsNl0sWzEwLDExXSxbMTEsNl0sWzcsNF0sWzExLDEyXSxbOSwxMl1d
\begin{tikzcd}[sep=scriptsize]
	{S_0} & {E_1} & {I_1} & {H_1} & {ICU_1} \\
	&& {R_1} && D \\
	{S_2} & {E_{2}} & {I_2} & {H_2} & {ICU_2} \\
	&& {R_2}
	\arrow[from=1-1, to=1-2]
	\arrow[from=1-2, to=1-3]
	\arrow[from=1-3, to=2-3]
	\arrow[from=1-1, to=3-2]
	\arrow[from=3-2, to=3-3]
	\arrow[from=2-3, to=1-1]
	\arrow[from=2-3, to=3-2]
	\arrow[from=3-3, to=4-3]
	\arrow[from=4-3, to=3-1]
	\arrow[from=1-3, to=1-4]
	\arrow[from=1-4, to=1-5]
	\arrow[from=1-5, to=2-3]
	\arrow[from=1-4, to=2-3]
	\arrow[from=3-3, to=3-4]
	\arrow[from=3-4, to=4-3]
	\arrow[from=3-4, to=3-5]
	\arrow[from=3-5, to=4-3]
	\arrow[from=3-1, to=3-2]
	\arrow[from=3-5, to=2-5]
	\arrow[from=1-5, to=2-5]
\end{tikzcd}
    \caption{Caption}
    \label{ch_5:fig:full_model_diagram_compact}
\end{figure}

% More text

% \begin{figure}
%     \centering
% % https://q.uiver.app/?q=WzAsMjQsWzEsMiwiU18wIl0sWzIsMiwiRV8xIl0sWzMsMiwiSV8xIl0sWzUsMiwiUl8xIl0sWzMsMSwiSF8xIl0sWzMsMCwiSUNVXzEiXSxbNSwwLCJEXzEiXSxbMSwzLCJFXzIiXSxbMSw0LCJJXzIiXSxbNSwzLCJFX3sxMn0iXSxbNSw0LCJJX3sxMn0iXSxbMyw1LCJSXzIiXSxbNiw0LCJIX3sxMn0iXSxbNiw1LCJJQ1VfezEyfSJdLFs2LDYsIkRfezEyfSJdLFswLDQsIkhfezJ9Il0sWzAsNSwiSUNVXzIiXSxbMCw2LCJEXzIiXSxbMyw2LCJTX3syMn0iXSxbMyw3LCJFX3syMn0iXSxbMyw4LCJJX3syMn0iXSxbNCw4LCJIX3syMn0iXSxbNSw4LCJJQ1VfezIyfSJdLFs2LDgsIkRfezIyfSJdLFswLDFdLFsxLDJdLFsyLDNdLFsyLDRdLFs0LDVdLFs1LDZdLFs1LDNdLFs0LDNdLFszLDAsIiIsMCx7ImN1cnZlIjotNX1dLFswLDddLFs4LDExXSxbMTAsMTFdLFs5LDEwXSxbMyw5XSxbNyw4XSxbMTAsMTJdLFsxMiwxM10sWzEzLDExXSxbMTIsMTFdLFsxMywxNF0sWzgsMTVdLFsxNSwxNl0sWzE2LDE3XSxbMTUsMTFdLFsxNiwxMV0sWzExLDE4XSxbMjAsMTEsIiIsMCx7ImN1cnZlIjotNX1dLFsxOCwxOV0sWzE5LDIwXSxbMjAsMjFdLFsyMSwyMl0sWzIyLDIzXSxbMjEsMTFdLFsyMiwxMV1d
% \begin{tikzcd}[sep=scriptsize]
% 	&&& {ICU_1} && {D_1} \\
% 	&&& {H_1} \\
% 	& {S_0} & {E_1} & {I_1} && {R_1} \\
% 	& {E_2} &&&& {E_{12}} \\
% 	{H_{2}} & {I_2} &&&& {I_{12}} & {H_{12}} \\
% 	{ICU_2} &&& {R_2} &&& {ICU_{12}} \\
% 	{D_2} &&& {S_{22}} &&& {D_{12}} \\
% 	&&& {E_{22}} \\
% 	&&& {I_{22}} & {H_{22}} & {ICU_{22}} & {D_{22}}
% 	\arrow[from=3-2, to=3-3]
% 	\arrow[from=3-3, to=3-4]
% 	\arrow[from=3-4, to=3-6]
% 	\arrow[from=3-4, to=2-4]
% 	\arrow[from=2-4, to=1-4]
% 	\arrow[from=1-4, to=1-6]
% 	\arrow[from=1-4, to=3-6]
% 	\arrow[from=2-4, to=3-6]
% 	\arrow[curve={height=-30pt}, from=3-6, to=3-2]
% 	\arrow[from=3-2, to=4-2]
% 	\arrow[from=5-2, to=6-4]
% 	\arrow[from=5-6, to=6-4]
% 	\arrow[from=4-6, to=5-6]
% 	\arrow[from=3-6, to=4-6]
% 	\arrow[from=4-2, to=5-2]
% 	\arrow[from=5-6, to=5-7]
% 	\arrow[from=5-7, to=6-7]
% 	\arrow[from=6-7, to=6-4]
% 	\arrow[from=5-7, to=6-4]
% 	\arrow[from=6-7, to=7-7]
% 	\arrow[from=5-2, to=5-1]
% 	\arrow[from=5-1, to=6-1]
% 	\arrow[from=6-1, to=7-1]
% 	\arrow[from=5-1, to=6-4]
% 	\arrow[from=6-1, to=6-4]
% 	\arrow[from=6-4, to=7-4]
% 	\arrow[curve={height=-30pt}, from=9-4, to=6-4]
% 	\arrow[from=7-4, to=8-4]
% 	\arrow[from=8-4, to=9-4]
% 	\arrow[from=9-4, to=9-5]
% 	\arrow[from=9-5, to=9-6]
% 	\arrow[from=9-6, to=9-7]
% 	\arrow[from=9-5, to=6-4]
% 	\arrow[from=9-6, to=6-4]
% \end{tikzcd}
%     \caption{Caption}
%     \label{ch_5:fig:full_model_diagram_sprawl}
% \end{figure}

\subsection{Application to California data}
\label{ch_5:subsec:application}

In our application, we focus on modeling the wave of cases, hospitalizations, ICU admissions, and deaths associated with the first Omicron variant of SARS-CoV-2 in California.
This wave lasted from roughly December 2021 to March 2022, but we use data beginning in May 2021 to forecast this time period.
We fit models to both Orange County data and statewide data.
The time series of daily counts of positive tests (cases), hospital occupancy and ICU occupancy are provided by the California Department of Public of Health, and published on the California Open Data Portal (\url{https://data.ca.gov}).
Additionally, the daily counts of sequenced viruses, aggregated by pango lineage \citep{pango}, are provided by the Global Initiative on Sharing All Influenza Data (GISAID) \citep{shu2017gisaid} and made available via Outbreak.info \citep{Gangavarapu2023}.
To match the format described in Section~\ref{ch_5:sec:methods}, the lineages are further aggregated into those that begin with ``BA.1" (e.g. BA.1, BA.1.1, BA.1.1.1, BA.1.17.2) and those that do not.
The non-genetic time series are aggregated at a weekly level, while the genetic data is used at a daily resolution.

\subsection{Discussion}
\label{ch_5:subsec:discussion}